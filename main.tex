\documentclass[12pt,a4paper]{article}
\usepackage[utf8]{inputenc}
\usepackage{dsfont} 
\usepackage[polish]{babel}
\usepackage{amsmath}
\usepackage{graphicx}
\usepackage[top=1in, bottom=1.5in, left=1.25in, right=1.25in]{geometry}
\usepackage{subfig}
\usepackage{multirow}
\usepackage{multicol}
\graphicspath{{Imagens/}}
\usepackage{xcolor,colortbl}
\usepackage{float}
\usepackage{fancyhdr} % Required for custom headers
\usepackage{lastpage} % Required to determine the last page for the footer
\usepackage{extramarks} % Required for headers and footers
\usepackage{indentfirst}
\usepackage{placeins}
\usepackage{scalefnt}
\usepackage{xcolor,listings}
\usepackage{textcomp}
\usepackage{color}
\usepackage{verbatim}
\usepackage{framed}
\usepackage{hyperref}

\definecolor{codegreen}{rgb}{0,0.6,0}
\definecolor{codegray}{rgb}{0.5,0.5,0.5}
\definecolor{codepurple}{HTML}{C42043}
\definecolor{backcolour}{HTML}{F2F2F2}
\definecolor{bookColor}{cmyk}{0,0,0,0.90}  
\color{bookColor}

\lstset{upquote=true}

\lstdefinestyle{mystyle}{
	backgroundcolor=\color{backcolour},   
	commentstyle=\color{codegreen},
	keywordstyle=\color{codepurple},
	numberstyle=\numberstyle,
	stringstyle=\color{codepurple},
	basicstyle=\footnotesize\ttfamily,
	breakatwhitespace=false,
	breaklines=true,
	captionpos=b,
	keepspaces=true,
	numbers=left,
	numbersep=10pt,
	showspaces=false,
	showstringspaces=false,
	showtabs=false,
}
\lstset{style=mystyle}

\newcommand\numberstyle[1]{%
	\footnotesize
	\color{codegray}%
	\ttfamily
	\ifnum#1<10 0\fi#1 |%
}

\definecolor{shadecolor}{HTML}{F2F2F2}

\newenvironment{sqltable}%
{\snugshade\verbatim}%
{\endverbatim\endsnugshade}

% Margins
\addtolength{\footskip}{0cm}
\addtolength{\textwidth}{1.4cm}
\addtolength{\oddsidemargin}{-.7cm}
\addtolength{\textheight}{1.6cm}
%\addtolength{\topmargin}{-2cm}

% paragrafo
\addtolength{\parskip}{.2cm}

% Set up the header and footer
\pagestyle{fancy}
\rhead{\hmwkAuthorName} % Top left header
\lhead{\hmwkClass: \hmwkTitle} % Top center header
\rhead{\firstxmark} % Top right header
\lfoot{Dominika Jelińska} % Bottom left footer
\cfoot{} % Bottom center footer
\rfoot{} % Bottom right footer
\renewcommand{\headrulewidth}{1pt}
\renewcommand{\footrulewidth}{1pt}

\newcommand{\hmwkTitle}{Implementacja sklepu internetowego opartego na Kubernetes}
\newcommand{\hmwkDueDate}{\today} 
\newcommand{\hmwkClass}{Technologie chmurowe}
\newcommand{\hmwkAuthorName}{Dominika Jelińska}

% trabalho 
\begin{document}
% capa
\begin{titlepage}
    \vfill
	\begin{center}
	\hspace*{-1cm}
	\vspace*{0.5cm}
    \includegraphics[scale=0.55]{imagens/loga.png}\\
	\textbf{Uniwersytet Gdański \\ [0.05cm]Wydział Matematyki, Fizyki i Informatyki \\ [0.05cm] Instytut Informatyki}

	\vspace{0.6cm}
	\vspace{4cm}
	{\huge \textbf{\hmwkTitle}}\vspace{8mm}
	
	{\large \textbf{\hmwkAuthorName}}\\[3cm]
	
		\hspace{.45\textwidth} %posiciona a minipage
	   \begin{minipage}{.5\textwidth}
	   Projekt z przedmiotu technologie chmurowe na kierunku informatyka profil praktyczny na Uniwersytecie Gdańskim.\\[0.1cm]
	  \end{minipage}
	  \vfill
	%\vspace{2cm}
	
	\textbf{Gdańsk}
	
	\textbf{\hmwkDueDate}
	\end{center}
	
\end{titlepage}

\newpage
\setcounter{secnumdepth}{5}
\tableofcontents
\newpage

\section{Opis projektu}
\label{sec:Project}

Projekt jest aplikacją internetową, która umożliwia użytkownikom przeglądanie, wyszukiwanie i zakup produktów online. Aplikacja obsługuje zarówno klientów, którzy mogą przeglądać produkty i dodawać je do koszyka, jak i zalogowanych użytkowników, którzy mogą dokonywać zakupów, zarządzać swoimi zamówieniami oraz zarządzać swoim profilem.

Projekt wykorzystuje kontenery Dockerowe do izolacji aplikacji oraz zarządzania jej środowiskiem uruchomieniowym. Kubernetes zapewnia automatyzację procesów wdrażania, aktualizacji oraz skalowania aplikacji.

\subsection{Opis architektury}
\label{sec:introduction}
Architektura aplikacji opartej na Kubernetes składa się z mikroserwisów działających w klastrze. Główne komponenty architektury obejmują:

\begin{enumerate}
  \item\textbf{Frontend}: Serwis odpowiedzialny za interakcję z użytkownikami poprzez przeglądarkę internetową. Serwis korzysta z Keycloak do autoryzacji użytkowników.

  \item\textbf{Backend}: API oraz logika biznesowa aplikacji, obsługująca żądania frontendu i komunikująca się z bazą danych.

  \item\textbf{Baza danych}: Baza MongoDB przechowująca informacje o produktach, zamówieniach oraz autoryzowanych użytkownikach.

  \item\textbf{Keycloak}: Usługa do zarządzania tożsamościami i dostępem do funkcjonalności aplikacji. Keycloak zapewnia bezpieczne uwierzytelnianie i autoryzację użytkowników.
\end{enumerate}

Architektura opiera się na mikroserwisach, co umożliwia niezależne skalowanie poszczególnych komponentów oraz elastyczność w zarządzaniu zasobami aplikacji.

Każdy z komponentów jest uruchamiany w osobnych kontenerach Dockerowych, co zapewnia izolację, niezależność i elastyczność w zarządzaniu zasobami. Kubernetes zarządza klastrami tych kontenerów, umożliwiając automatyczne skalowanie za pomocą replik.

\subsection{Opis infrastruktury}
\label{sec:Users}
Aplikacja działa w środowisku Kubernetes, uruchomionym lokalnie za pomocą Minikube. Minikube umożliwia łatwe emulowanie klastra Kubernetes na pojedynczej maszynie, co jest idealnym rozwiązaniem do testowania i rozwoju aplikacji przed jej wdrożeniem na środowisko produkcyjne. 

W klastrze Kubernetes zaimplementowane są trzy główne komponenty: frontend obsługiwany przez serwis Frontend-Service, backend przez Backend-Service, oraz baza danych MongoDB zarządzana przez serwis Mongo-Service. Frontend-Service udostępnia interfejs użytkownika, komunikujący się z backendem przez Backend-Service, który obsługuje logikę biznesową aplikacji. MongoDB zapewnia przechowywanie danych o użytkownikach, produktach i zamówieniach.

Konfiguracja sieci w klastrze Kubernetes oraz odpowiednie przypisanie zasobów, takich jak limit wykorzystywanej pamięci RAM są dostosowane do potrzeb każdego z komponentów aplikacji.

\subsection{Opis komponentów aplikacji}
\label{sec:FunctionalConditions}
Poniżej przedstawiono szczegółowy opis poszczególnych komponentów:

\begin{enumerate}
    \item\textbf{Frontend} aplikacji jest uruchamiany jako aplikacja internetowa w kontenerze Dockerowym. Komunikuje się z backendem za pomocą interfejsu API. Dzięki skonfigurowanemu deploymentowi i serwisowi w Kubernetes, frontend jest łatwo skalowalny i dostępny dla użytkowników. Frontend integruje się również z usługą Keycloak dla autoryzacji użytkowników. Konfiguracja frontendu, taka jak adresy URL backendu i Keycloak są przekazywane przez ConfigMap w Kubernetes.

    \item\textbf{Keycloak} jest usługą uruchamianą jako deployment w kontenerze Dockerowym. Dzięki skonfigurowanemu serwisowi, Keycloak obsługuje żądania uwierzytelniania od frontendu i backendu. Konfiguracja Keycloak, taka jak ustawienia klienta aplikacji, jest przekazywana przez ConfigMap.

    \item\textbf{MongoDB} jest wykorzystywana jako baza danych NoSQL dla produktów, zamówień i profili użytkowników w aplikacji. Jest uruchomiona jako deployment w kontenerze Dockerowym. Poprzez skonfigurowany serwis w Kubernetes, umożliwia backendowi dostęp do przechowywanych danych.

    \item\textbf{Backend} aplikacji jest ruchamiany jako mikroserwis w kontenerze Dockerowym. Odpowiada za logikę biznesową i komunikację z MongoDB oraz frontendem przez interfejs API. Dzięki deploymentowi i serwisowi w Kubernetes, backend zapewnia dostępność danych i może być skalowalny. Backend integruje się również z Keycloak dla bezpiecznej autoryzacji użytkowników. Informacje konfiguracyjne, takie jak adresy URL MongoDB i ustawienia integracji z Keycloak, są przekazywane przez ConfigMap.
\end{enumerate}

\subsection{Konfiguracja i zarządzanie}
\label{sec:NonFunctionalConditions}
Każdy komponent aplikacji jest uruchamiany jako kontener Dockerowy, co ułatwia ich przenoszenie między środowiskami deweloperskimi a produkcyjnymi.

Ustawienia konfiguracyjne, takie jak adresy URL do serwisów backendowych, parametry połączenia z bazą danych oraz konfiguracja Keycloak, są przekazywane do kontenerów za pomocą ConfigMap w Kubernetes. Dzięki temu zmiany w konfiguracji mogą być łatwo wprowadzane i zarządzane bez konieczności modyfikowania obrazów kontenerów.

\subsection{Zarządzanie błędami}
\label{sec:ERD} 
Do monitorowania aplikacji można wykorzystać narzędzie dostępne w Kubernetes, takie jak Kubernetes Dashboard.

\subsection{Skalowalność}
\label{sec:ExamplesSection}
Zarówno frontend, jak i backend aplikacji są uruchamiane w Kubernetes przy użyciu replik, dzięki czemu można łatwo zwiększać lub zmniejszać liczbę instancji aplikacji. Skalować można poprzez modyfikację liczby replik bezpośrednio w pliku konfiguracyjnym lub za pomocą polecenia \texttt{kubectl}.

Obecnie MongoDB oraz Keycloak są skonfigurowane do działania z pojedynczą instancją. To podejście minimalizuje skomplikowanie dostępu i zarządzania danymi w systemie.

\subsection{Wymagania dotyczące zasobów}
\label{sec:ExampleTables}
Kontener frontendu aplikacji został skonfigurowany z określonymi zasobami pamięci RAM. Konfiguracja obejmuje ustawienia requests i limits.

\begin{itemize}
    \item\textbf{requests}: Określa minimalne wymagane zasoby, które kontener potrzebuje do poprawnego działania. W przypadku frontendu wynosi to 256 MB pamięci RAM.
    
    \item\textbf{limits}: Określa maksymalną ilość pamięci RAM, jaką kontener może zużyć. Dla frontendu ustalono limit na 512 MB pamięci RAM.
\end{itemize}

\subsection{Architektura sieciowa}
\label{sec:ExampleResults}
Aplikacja wykorzystuje różne typy serwisów do zapewnienia komunikacji między komponentami oraz dostępu dla użytkowników zewnętrznych.

\begin{itemize}
    \item\textbf{Serwisy typu ClusterIP}: Wewnętrzne komponenty aplikacji komunikują się ze sobą za pomocą serwisów typu ClusterIP. Każdy serwis posiada stały adres IP wewnątrz klastra, co umożliwia bezpieczną komunikację między pods.

    \item\textbf{Serwis typu NodePort dla frontendu}: Frontend aplikacji jest wystawiony na zewnątrz klastra za pomocą serwisu typu NodePort. Jest to sposób umożliwiający dostęp do aplikacji z zewnątrz poprzez publiczny adres IP węzła klastra i wybrany port.
\end{itemize}

\begin{thebibliography}{9}
\bibitem{kubernetes-docs}
Dokumentacja Kubernetes
\url{https://kubernetes.io/pl/docs/concepts/overview/}

\end{thebibliography}

\end{document}
